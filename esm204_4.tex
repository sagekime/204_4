% Options for packages loaded elsewhere
\PassOptionsToPackage{unicode}{hyperref}
\PassOptionsToPackage{hyphens}{url}
%
\documentclass[
]{article}
\usepackage{lmodern}
\usepackage{amssymb,amsmath}
\usepackage{ifxetex,ifluatex}
\ifnum 0\ifxetex 1\fi\ifluatex 1\fi=0 % if pdftex
  \usepackage[T1]{fontenc}
  \usepackage[utf8]{inputenc}
  \usepackage{textcomp} % provide euro and other symbols
\else % if luatex or xetex
  \usepackage{unicode-math}
  \defaultfontfeatures{Scale=MatchLowercase}
  \defaultfontfeatures[\rmfamily]{Ligatures=TeX,Scale=1}
\fi
% Use upquote if available, for straight quotes in verbatim environments
\IfFileExists{upquote.sty}{\usepackage{upquote}}{}
\IfFileExists{microtype.sty}{% use microtype if available
  \usepackage[]{microtype}
  \UseMicrotypeSet[protrusion]{basicmath} % disable protrusion for tt fonts
}{}
\makeatletter
\@ifundefined{KOMAClassName}{% if non-KOMA class
  \IfFileExists{parskip.sty}{%
    \usepackage{parskip}
  }{% else
    \setlength{\parindent}{0pt}
    \setlength{\parskip}{6pt plus 2pt minus 1pt}}
}{% if KOMA class
  \KOMAoptions{parskip=half}}
\makeatother
\usepackage{xcolor}
\IfFileExists{xurl.sty}{\usepackage{xurl}}{} % add URL line breaks if available
\IfFileExists{bookmark.sty}{\usepackage{bookmark}}{\usepackage{hyperref}}
\hypersetup{
  pdftitle={ESM 204 \#4},
  pdfauthor={Sage Kime and Karla Garibay Garcia and Craig Stuart},
  hidelinks,
  pdfcreator={LaTeX via pandoc}}
\urlstyle{same} % disable monospaced font for URLs
\usepackage[margin=1in]{geometry}
\usepackage{color}
\usepackage{fancyvrb}
\newcommand{\VerbBar}{|}
\newcommand{\VERB}{\Verb[commandchars=\\\{\}]}
\DefineVerbatimEnvironment{Highlighting}{Verbatim}{commandchars=\\\{\}}
% Add ',fontsize=\small' for more characters per line
\usepackage{framed}
\definecolor{shadecolor}{RGB}{248,248,248}
\newenvironment{Shaded}{\begin{snugshade}}{\end{snugshade}}
\newcommand{\AlertTok}[1]{\textcolor[rgb]{0.94,0.16,0.16}{#1}}
\newcommand{\AnnotationTok}[1]{\textcolor[rgb]{0.56,0.35,0.01}{\textbf{\textit{#1}}}}
\newcommand{\AttributeTok}[1]{\textcolor[rgb]{0.77,0.63,0.00}{#1}}
\newcommand{\BaseNTok}[1]{\textcolor[rgb]{0.00,0.00,0.81}{#1}}
\newcommand{\BuiltInTok}[1]{#1}
\newcommand{\CharTok}[1]{\textcolor[rgb]{0.31,0.60,0.02}{#1}}
\newcommand{\CommentTok}[1]{\textcolor[rgb]{0.56,0.35,0.01}{\textit{#1}}}
\newcommand{\CommentVarTok}[1]{\textcolor[rgb]{0.56,0.35,0.01}{\textbf{\textit{#1}}}}
\newcommand{\ConstantTok}[1]{\textcolor[rgb]{0.00,0.00,0.00}{#1}}
\newcommand{\ControlFlowTok}[1]{\textcolor[rgb]{0.13,0.29,0.53}{\textbf{#1}}}
\newcommand{\DataTypeTok}[1]{\textcolor[rgb]{0.13,0.29,0.53}{#1}}
\newcommand{\DecValTok}[1]{\textcolor[rgb]{0.00,0.00,0.81}{#1}}
\newcommand{\DocumentationTok}[1]{\textcolor[rgb]{0.56,0.35,0.01}{\textbf{\textit{#1}}}}
\newcommand{\ErrorTok}[1]{\textcolor[rgb]{0.64,0.00,0.00}{\textbf{#1}}}
\newcommand{\ExtensionTok}[1]{#1}
\newcommand{\FloatTok}[1]{\textcolor[rgb]{0.00,0.00,0.81}{#1}}
\newcommand{\FunctionTok}[1]{\textcolor[rgb]{0.00,0.00,0.00}{#1}}
\newcommand{\ImportTok}[1]{#1}
\newcommand{\InformationTok}[1]{\textcolor[rgb]{0.56,0.35,0.01}{\textbf{\textit{#1}}}}
\newcommand{\KeywordTok}[1]{\textcolor[rgb]{0.13,0.29,0.53}{\textbf{#1}}}
\newcommand{\NormalTok}[1]{#1}
\newcommand{\OperatorTok}[1]{\textcolor[rgb]{0.81,0.36,0.00}{\textbf{#1}}}
\newcommand{\OtherTok}[1]{\textcolor[rgb]{0.56,0.35,0.01}{#1}}
\newcommand{\PreprocessorTok}[1]{\textcolor[rgb]{0.56,0.35,0.01}{\textit{#1}}}
\newcommand{\RegionMarkerTok}[1]{#1}
\newcommand{\SpecialCharTok}[1]{\textcolor[rgb]{0.00,0.00,0.00}{#1}}
\newcommand{\SpecialStringTok}[1]{\textcolor[rgb]{0.31,0.60,0.02}{#1}}
\newcommand{\StringTok}[1]{\textcolor[rgb]{0.31,0.60,0.02}{#1}}
\newcommand{\VariableTok}[1]{\textcolor[rgb]{0.00,0.00,0.00}{#1}}
\newcommand{\VerbatimStringTok}[1]{\textcolor[rgb]{0.31,0.60,0.02}{#1}}
\newcommand{\WarningTok}[1]{\textcolor[rgb]{0.56,0.35,0.01}{\textbf{\textit{#1}}}}
\usepackage{graphicx,grffile}
\makeatletter
\def\maxwidth{\ifdim\Gin@nat@width>\linewidth\linewidth\else\Gin@nat@width\fi}
\def\maxheight{\ifdim\Gin@nat@height>\textheight\textheight\else\Gin@nat@height\fi}
\makeatother
% Scale images if necessary, so that they will not overflow the page
% margins by default, and it is still possible to overwrite the defaults
% using explicit options in \includegraphics[width, height, ...]{}
\setkeys{Gin}{width=\maxwidth,height=\maxheight,keepaspectratio}
% Set default figure placement to htbp
\makeatletter
\def\fps@figure{htbp}
\makeatother
\setlength{\emergencystretch}{3em} % prevent overfull lines
\providecommand{\tightlist}{%
  \setlength{\itemsep}{0pt}\setlength{\parskip}{0pt}}
\setcounter{secnumdepth}{-\maxdimen} % remove section numbering

\title{ESM 204 \#4}
\author{Sage Kime and Karla Garibay Garcia and Craig Stuart}
\date{5/26/2021}

\begin{document}
\maketitle

\hypertarget{quadratic-damage-function}{%
\subsubsection{1. Quadratic Damage
Function}\label{quadratic-damage-function}}

Using damages.csv, estimate a quadratic damage function relating the
dollar value of damages to the change in global mean temperature. Omit
an intercept term; damages by construction must equal zero when there is
no climate change. Plot your estimated damage function, overlaid with a
scatterplot of the underlying data.

\includegraphics{esm204_4_files/figure-latex/unnamed-chunk-2-1.pdf}

\textbf{Quadratic Damage Function: y = -1.035e12x + 1.94e13x\^{}2}

\hypertarget{damages-in-each-year-under-the-baseline-climate-and-pulse-scenarios}{%
\subsubsection{2. Damages in each year under the baseline climate and
pulse
scenarios}\label{damages-in-each-year-under-the-baseline-climate-and-pulse-scenarios}}

Use warming.csv and your estimated damage function to predict damages in
each year under the baseline climate and the pulse scenario. Make four
plots: (1) damages over time without the pulse, (2) damages over time
with the pulse, (3) the difference in damages over time that arises from
the pulse, and (4) the difference in damages over time from the pulse
per ton of CO2 (you can assume that each ton of the pulse causes the
same amount of damage).

\includegraphics{esm204_4_files/figure-latex/unnamed-chunk-4-1.pdf}
\includegraphics{esm204_4_files/figure-latex/unnamed-chunk-4-2.pdf}
\includegraphics{esm204_4_files/figure-latex/unnamed-chunk-4-3.pdf}
\includegraphics{esm204_4_files/figure-latex/unnamed-chunk-4-4.pdf}

\hypertarget{calculate-the-scc}{%
\subsubsection{3. Calculate the SCC}\label{calculate-the-scc}}

The SCC is the present discounted value of the stream of future damages
caused by one additional ton of CO2. The Obama Administration used a
discount rate of 3\% to discount damages. Recently, New York State used
a discount rate of 2\%. Calculate and make a plot of the SCC (y-axis)
against the discount rate (x-axis) for a reasonable range of discount
rates.

\includegraphics{esm204_4_files/figure-latex/unnamed-chunk-5-1.pdf}

\hypertarget{ramseys-rule}{%
\subsubsection{4. Ramsey's Rule}\label{ramseys-rule}}

The National Academies of Sciences, Engineering, and Medicine advised
the government in a 2017 report to use the Ramsey Rule when discounting
within the SCC calculation: r = ρ + ηg Using ρ = 0.001, η = 2, and g =
0.01, what is the SCC? Locate this point on your graph from above.

\begin{Shaded}
\begin{Highlighting}[]
\CommentTok{#Ramsey's Rule: discount rate}
\NormalTok{r <-}\StringTok{ }\FloatTok{.001} \OperatorTok{+}\StringTok{ }\DecValTok{2}\OperatorTok{*}\NormalTok{.}\DecValTok{01}

\CommentTok{#SCC at discount rate of .021 (Ramsey's Rule)}
\KeywordTok{PV_calc}\NormalTok{(warming}\OperatorTok{$}\NormalTok{damages_ton, }\FloatTok{.021}\NormalTok{)}
\end{Highlighting}
\end{Shaded}

\begin{verbatim}
## [1] 70.65026
\end{verbatim}

\hypertarget{policy-a-vs.-policy-b}{%
\subsubsection{5. Policy A vs.~Policy B}\label{policy-a-vs.-policy-b}}

What is the expected present value of damages up to 2100 under Policy A?

\begin{Shaded}
\begin{Highlighting}[]
\CommentTok{#Policy A}

\CommentTok{#Calculate PV using ~for loop~}
\NormalTok{PV_calc <-}\StringTok{ }\ControlFlowTok{function}\NormalTok{(values, discount_rate) \{}
\NormalTok{  sum <-}\StringTok{ }\DecValTok{0}\NormalTok{;}
\NormalTok{  r <-}\StringTok{ }\NormalTok{discount_rate}
  \ControlFlowTok{for}\NormalTok{ (i }\ControlFlowTok{in} \KeywordTok{c}\NormalTok{(}\DecValTok{1}\OperatorTok{:}\KeywordTok{length}\NormalTok{(values))) \{}
\NormalTok{    current <-}\StringTok{ }\NormalTok{values[i]}\OperatorTok{/}\NormalTok{(}\DecValTok{1} \OperatorTok{+}\StringTok{ }\NormalTok{r)}\OperatorTok{^}\NormalTok{i}
\NormalTok{    sum <-}\StringTok{ }\NormalTok{sum }\OperatorTok{+}\StringTok{ }\NormalTok{current}
\NormalTok{  \}}
  \KeywordTok{return}\NormalTok{(sum)}
\NormalTok{\}}

\CommentTok{#Create new variable for 1.5x baseline in policy A}
\NormalTok{warming}\OperatorTok{$}\NormalTok{baseline150 <-}\StringTok{ }\NormalTok{warming}\OperatorTok{$}\NormalTok{warming_baseline}\OperatorTok{*}\FloatTok{1.5}

\CommentTok{#Create new variable for damages under 1.5 baseline outcome}
\NormalTok{warming}\OperatorTok{$}\NormalTok{damages_baseline150 <-}\StringTok{ }\KeywordTok{damages_model}\NormalTok{(warming}\OperatorTok{$}\NormalTok{baseline150)}

\CommentTok{#Calculate PV of each outcome in policy A}
\NormalTok{A_baseline <-}\StringTok{ }\KeywordTok{PV_calc}\NormalTok{(warming}\OperatorTok{$}\NormalTok{damages_baseline, }\FloatTok{0.02}\NormalTok{)}
\NormalTok{A_baseline150 <-}\StringTok{ }\KeywordTok{PV_calc}\NormalTok{(warming}\OperatorTok{$}\NormalTok{damages_baseline150, }\FloatTok{0.02}\NormalTok{)}

\NormalTok{expected_value <-}\StringTok{ }\FloatTok{.5}\OperatorTok{*}\NormalTok{A_baseline }\OperatorTok{+}\StringTok{ }\FloatTok{.5}\OperatorTok{*}\NormalTok{A_baseline150}
\NormalTok{expected_value}
\end{Highlighting}
\end{Shaded}

\begin{verbatim}
## [1] 3.034025e+15
\end{verbatim}

What is the expected present value of damages up to 2100 under Policy B?

\begin{Shaded}
\begin{Highlighting}[]
\CommentTok{#Policy B}

\CommentTok{#Create new variable for warming until 2050}
\NormalTok{warming <-}\StringTok{ }\NormalTok{warming }\OperatorTok\StringTok{ }
\StringTok{  }\KeywordTok{mutate}\NormalTok{(}\DataTypeTok{warming_baseline_2050 =} 
           \KeywordTok{case_when}\NormalTok{(year }\OperatorTok{>}\StringTok{ }\DecValTok{2050} \OperatorTok{~}\StringTok{ }\FloatTok{1.29}\NormalTok{,}
                     \OtherTok{TRUE} \OperatorTok{~}\StringTok{ }\NormalTok{warming_baseline}
\NormalTok{          ))}

\CommentTok{#Create new variable for damages under 2050 baseline outcome}
\NormalTok{warming}\OperatorTok{$}\NormalTok{damages_baseline2050 <-}\StringTok{ }\KeywordTok{damages_model}\NormalTok{(warming}\OperatorTok{$}\NormalTok{warming_baseline_}\DecValTok{2050}\NormalTok{)}

\CommentTok{#Calculate PV of policy B}
\NormalTok{B_baseline <-}\StringTok{ }\KeywordTok{PV_calc}\NormalTok{(warming}\OperatorTok{$}\NormalTok{damages_baseline2050, }\FloatTok{0.02}\NormalTok{) }\CommentTok{#PV is expected value bc only one option}
\NormalTok{B_baseline}
\end{Highlighting}
\end{Shaded}

\begin{verbatim}
## [1] 7.782494e+14
\end{verbatim}

Suppose undertaking Policy A costs zero and undertaking Policy B costs
X. How large could X be for it to still make economic sense to pursue
Policy B instead of Policy A?

\begin{Shaded}
\begin{Highlighting}[]
\CommentTok{#Subtract cost of policy A - policy B}
\NormalTok{max_x <-}\StringTok{ }\NormalTok{expected_value }\OperatorTok{-}\StringTok{ }\NormalTok{B_baseline}
\NormalTok{max_x}
\end{Highlighting}
\end{Shaded}

\begin{verbatim}
## [1] 2.255776e+15
\end{verbatim}

Qualitatively, how would your answer change if society were risk averse?

In the risk neutral society when the cost of both policies are equal,
the expected utility of Policy A would be equal to the utility of Policy
B. When society is risk averse, the utility of Policy B is greater than
the expected utility of Policy A, therefore society would be more likely
to choose Policy B.

\end{document}
